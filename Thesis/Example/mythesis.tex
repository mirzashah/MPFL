%% thesis.tex
%%
%% this file, mythesis.tex, is the main file of a fictitious
%% Penn State Ph D thesis 
%%
%% 
%% this material can be used as a template to prepare your own Ph D thesis
%%
%% this file was created Sept 1995 by Stephen G. Simpson,
%% simpson@math.psu.edu
%%
%% revised November 1996, S. Simpson
%% revised 2002, Sarah Gallager (to allow deluxetables)
%% modified a little more by Michele Stark (2004)
%% modified to use the new psuthesis.cls (Mar 2005) with a signature 
%%   page and a committee page

\documentclass[11pt]{psuthesis}

%% optional packages, in case you want AMS math macros and AMS symbols
\usepackage{amsmath,amssymb}
%% allows bibtex, \citet{}, \citep{} referencing:
\usepackage{natbib}
%% I truthfully don't know what the following is for, but I never used it:
%\citestyle{aa}
%% optional package, in case you want PostScript graphics:
%%\usepackage{psfig,graphics} 
%% the following allows you to use the AAS deluxetable environment:
\usepackage{deluxetable}
%% Use the following if you have tables that are longer than a single page::
%\usepackage{longtable}
%% Use the following if you have (non-deluxetables, i.e., longtables) that
%% you need to display landscape oriented:
%\usepackage{lscape}

%% for a less-than-final version of the thesis, this command
%% places "DRAFT: <date> AT <time>" at the top of each page...
%\thesisdraft    
%% (comment this out for the final version)

%% you can speed things up by compiling only one chapter at a time
%\includeonly{somechapter}
%\includeonly{someotherchapter}
%\includeonly{yetanotherchapter}
%\includeonly{Ithinkyougettheideachapter}
%\includeonly{conclusions}
%% (comment this out for the final version)

%% Fix the text citations so that there is no comma between the authors and 
%% year.  This will help contain the furious Brandt red pen.
\bibpunct{(}{)}{;}{a}{}{,}

%% Change the fonts back to something reasonable.
%% Note: scriptsize is typically smaller than footnotesize.
%\renewcommand{\scriptsize{\@setfontsize\scriptsize\@ixpt{9pt}}
%\renewcommand{\footnotesize{\@setfontsize\scriptsize\@xpt{10pt}}

%%
%% These are all the definitions that I've used throughout my thesis.
%%\input{/enter/path/to/your/definitions}

\begin{document}

%%Comment this out for the final thesis:

%%

%% at the beginning of the thesis we have a title page, a signature
%% page, and an abstract

\author{(Name Goes Here)}


\title{\uppercase {This is where the title goes: \\
       it can even be double lined!}}

\dept{Astronomy and Astrophysics}

\submitdate{May 2002}

\copyrightyear{2002}


\begin{singlespace}

\readerone{Name of Associate Professor Advisor \\
         \assocprof{Astronomy and Astrophysics} \\
         \adviser \\
         \chair}

\readertwo{Associate Prof.\ Member1\\
           \assocprof{Astronomy and Astrophysics}}

\readerthree{Assistant Prof.\ Member2 \\
             \assistprof{Astronomy and Astrophysics}}

\readerfour{Associate Prof.\ Member3 \\
             \assocprof{Physics}}

\readerfive{Evan Pugh Prof.\ Member4 \\
             Evan Pugh \prof{Astronomy and Astrophysics}}

\readersix{Assistant Prof.\ Member5 \\
             \assistprof{Astronomy and Astrophysics}}

\readerseven{Professor and Department Head Member6 \\
             \prof{Astronomy and Astrophysics}\\
             \head{Astronomy and Astrophysics}}

%%   Key to titles:
%% Associate Professor: \asocprof{of what}
%% Assistant Professor: \assistprof{of what}
%% Full Professor: \prof{of what}
%% also Dept. Head: \head{of what}
%% You can also do things like: ``Associate \head{of what}''

\begin{frontmatter}

%this is the ``normal'' signature page from the original version of the class
\signaturepage

\begin{doublespace}
\titlepage
\end{doublespace}

%this is the new committee page
\committeepage

\abstract


Here is where the text of the Abstract goes.



%% this is the end of the abstract

%% after the abstract come the table of contents, the list of tables,
%% and the list of figures
%% Note about the figure list... so the figure list is a decent length,
%% use the following command for the figure captions:
%% \caption[Short figure title to appear in figure list]{Normal figure caption}
%% (this trick unfortunately does not work with the deluxetable 
%% ``\tablecaption'' command, so be careful what you put in the table captions)
%%
%% If you have really long tables that cover multiple pages, you
%% will want to use the ``longtable'' environment (it is very similer to
%% deluxetable) but allows you to specify headers and footers for the first,
%% last, and middle pages of the figure.

%% use this command to omit the list of tables,
%% if the thesis doesn't contain any tables
%\nolotables

%% use this command to omit the list of figures,
%% if the thesis doesn't contain any figures
%\nolofigures

\tables

%% next come the acknowlegements (optional) and the preface (optional)

\acknowledgments  % optional

Here is where the acknowledgments go if you have any.


%\preface    % optional

\clearpage

\vspace*{2.0truein}

\LARGE
\parbox{4.0truein}{
\par\noindent
Rome did not create a great empire by having meetings, they did it by
killing all who opposed them.\\
\hspace*{\fill}--Unknown
}
\normalsize

\end{frontmatter}

%% this is the end of the front matter

%% now we include the actual chapters of the thesis
%% there are individual chapter files ch-intr.tex, ch-over.tex, ...
%% (NOTE: you do not need the ``.tex'' extention on the file name 
%% in the include statement)
%% these chapters can be in a sub-directory, for example: 
%% \include{chapterdirectory/chaptername}

%\include{intro}

%\include{somechaper}

%\include{someotherchapter}

%\include{yetanotherchapter}

%\include{Ithinkyougettheideachapter}

%\include{conclusions}

%% now we include the appendices
%\appendices
%% omit this if there are no appendices

%% if there is only one appendix, 
%% say \singleappendix instead of \appendices
% \singleappendix

%% these are the actual commands to include the apendix files:

%\include{ap-addi}

%\include{ap-more}


%% finally comes the bibliography and vita
%% the bibliography is generated automatically using BibTeX
%% Note: in order to get the (author year) citation style, this example 
%% includes a different *.bst style file than psuthesis.bst.  Sarah found 
%% this style file on: http://www.ee.oulu.fi/~harza/latex/  
%% This style includes the paper titles in the bibliography, so use it 
%% if you want:
%\bibliographystyle{/enter/path/to/ayphdthesis_mod}
%% However, according to the grad school thesis guide (c.2004), there is no 
%% special formating required for the bibliography, and to just follow 
%% the citation styles of your field, in that case the ``apj'' style is 
%% prefectly OK, so that is the one that is going to be included in this
%% example file (use whichever style you prefer):
\bibliographystyle{apj}

%% the following is the path to you *.bib file 
%% (you do not need to enter the ``.bib'' extention)
\bibliography{/enter/path/to/your/bib/references/file}
%% ADS can generate the entries in the bib file for you, just call up
%% the abstract, then near the bottom of the abstract page there is a
%% link to ``Bibtex entry for this abstract'' - just copy that into 
%% your bib file.  Here is an example bibtex entry:
%% @ARTICLE{citecode,
%%    author = {{Smith}, J. and {Jones}, M.},
%%     title = "{This Paper has some Really Cool Results}",
%%   journal = {\aj},
%%      year = 2002,
%%     month = sep,
%%    volume = 123,
%%     pages = {1-20},
%%    adsurl = {http://adsabs.harvard.edu/cgi-bin/nph-bib_query?bibcode=2002AJ....123.1S&amp;db_key=AST},
%%   adsnote = {Provided by the NASA Astrophysics Data System}
%% }
%% NOTE: ADS returns the AASTeX code for the journal name, you'll either 
%% have to change that by hand in each bib entry, or include a 
%% definition of the codes in your ``definitions'' file so LaTeX doesn't 
%% freak out.  Also, what I entered as ``citecode'' can be changed to 
%% whatever you want to use in the citations in the document, i.e., for
%% ``\citet{citecode}'' or ``\citep{citecode}''.


%% the thesis must end with a Curriculum Vitae (**one page or less**)
%% (this is Sarah's formatting, not sure how it compares to the other examples)
\vita
\Large
\vspace*{-0.4truein}
\centerline{{\bf Your Name Here}}

\medskip

\large
\centerline{{\bf Education}}
\normalsize

\smallskip

\par\noindent
\textbf{\textit{The Pennsylvania State University}}\, State College, Pennsylvania\hfill 199?--Present

\smallskip

\par\noindent
\hspace{0.10truein}  
\parbox{6.15truein}{
\par\noindent
Ph.D. in Astronomy \& Astrophysics, expected in (Month) (Year) \\ Area
of Specialization: Something }

\medskip

\par\noindent
\textbf{\textit{(Undergrad Institute Name)}}\, (City), (State)\hfill 199?--199?

\smallskip

\par\noindent
\hspace{0.10truein}  
\parbox{6.15truein}{
\par\noindent
B.S. in Physics (or whatever), \textit{cum laude} with distinction in the major (etc.)
}

\medskip

\large
\centerline{{\bf Awards and Honors}}
\normalsize

\smallskip

\par\noindent
Award Name \hfill year\\
Another Award \hfill 200?--200?\\
One more for good measure \hfill 200?, 200?
%NASA Graduate Student Research Program Fellowship\hfill2000--Present\\ 
%Pennsylvania Space Grant Consortium Fellowship\hfill1998--2000, 2000--Present\\
%Zaccheus Daniel Foundation for Astronomical Science Grant\hfill1999, 2000\\
%University Graduate Fellowship\hfill1997--1998\\
%Roberts Fellowship, Eberly College of Science\hfill1997--1998\\
%Summer Study Grant, Holderness School\hfill1996

\medskip

\large
\centerline{{\bf Research Experience}}
\normalsize

\smallskip

\par\noindent
\textbf{\textit{Doctoral Research}}\, The Pennsylvania State University\hfill199?--Present
\par\noindent
Thesis Advisor: Prof. Someone R. Other

\smallskip

\par\noindent
\hspace{0.10truein}  
\parbox{5.7truein}{
\par\noindent
This research involved lots of cool stuff.
}

\medskip

\par\noindent
\textbf{\textit{Graduate Research}}\, The Pennsylvania State University\hfill 199?--199?
\par\noindent
Research Advisor: Prof. Someoneelse R. Other

\smallskip

\par\noindent
\hspace{0.10truein}  
\parbox{5.7truein}{
\par\noindent
This research also involved lots of cool stuff.
}

\medskip

\par\noindent
\textbf{\textit{Undergraduate Research}}\, Undergrad University\hfill 199?--199?
\par\noindent
Research Advisor: Prof. Name Goes Here

\smallskip

\par\noindent
\hspace{0.10truein}  
\parbox{5.7truein}{
\par\noindent
Really cool stuff is what I did for this research.
}

\medskip

\large
\centerline{{\bf Teaching Experience}}
\normalsize

\smallskip

\par\noindent
\textbf{\textit{Guest Lecturer}}\, The Pennsylvania State University\hfill 199?--Present

\smallskip

\par\noindent
\hspace{0.10truein}  
\parbox{5.7truein}{
\par\noindent
I taught lectures which involved doing cool stuff.
}

\medskip

\par\noindent
\textbf{\textit{Teaching Assistant}}\, The Pennsylvania State University\hfill 199?

\smallskip

\par\noindent
\hspace{0.10truein}  
\parbox{5.7truein}{
\par\noindent
I taught labs and did cool stuff.
}

\medskip

\par\noindent
\textbf{\textit{Another Experience}}\, Some School, Some City, Some State\hfill 199?--199?

\smallskip

\par\noindent
\hspace{0.10truein}  
\parbox{5.7truein}{
\par\noindent
I taught cool stuff to cool people.
}

\end{singlespace}
\end{document}

%%% Local Variables: 
%%% mode: latex
%%% TeX-master: t
%%% End: 
