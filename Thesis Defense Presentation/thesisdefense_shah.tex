 \documentclass{beamer}



\setbeamercolor{alerted text}{fg=white}
\setbeamercolor{background canvas}{bg=black}
\setbeamercolor{block body alerted}{bg=normal text.bg!90!black}
\setbeamercolor{block body}{bg=normal text.bg!90!black}
\setbeamercolor{block body example}{bg=normal text.bg!90!black}
\setbeamercolor{block title alerted}{use={normal text,alerted text},fg=alerted text.fg!75!normal text.fg,bg=normal text.bg!75!black}
\setbeamercolor{block title}{bg=blue}
\setbeamercolor{block title example}{use={normal text,example text},fg=example text.fg!75!normal text.fg,bg=normal text.bg!75!black}
\setbeamercolor{fine separation line}{}
\setbeamercolor{frametitle}{fg=white}
\setbeamercolor{item projected}{fg=black}
\setbeamercolor{normal text}{bg=black,fg=white}
\setbeamercolor{palette sidebar primary}{use=normal text,fg=normal text.fg}
\setbeamercolor{palette sidebar quaternary}{use=structure,fg=structure.fg}
\setbeamercolor{palette sidebar secondary}{use=structure,fg=structure.fg}
\setbeamercolor{palette sidebar tertiary}{use=normal text,fg=normal text.fg}
\setbeamercolor{section in sidebar}{fg=brown}
\setbeamercolor{section in sidebar shaded}{fg= grey}
\setbeamercolor{separation line}{}
\setbeamercolor{sidebar}{bg=red}
\setbeamercolor{sidebar}{parent=palette primary}
\setbeamercolor{structure}{bg=black, fg=red}
\setbeamercolor{subsection in sidebar}{fg=brown}
\setbeamercolor{subsection in sidebar shaded}{fg= grey}
\definecolor{darkgrey}{RGB}{103,103,103}
\setbeamercolor{title}{fg=darkgrey}
\setbeamercolor{titlelike}{fg=white}
\setbeamercolor{itemize item}{fg=darkgrey} % all frames will have red bullets
\setbeamertemplate{navigation symbols}{}%remove navigation symbols

\mode<presentation>
{
  \usetheme{default}
  
  % or ...

  \setbeamercovered{transparent}
  % or whatever (possibly just delete it)
}


\usepackage[english]{babel}
% or whatever

\usepackage[latin1]{inputenc}
% or whatever

\usepackage{times}
\usepackage[T1]{fontenc}
% Or whatever. Note that the encoding and the font should match. If T1
% does not look nice, try deleting the line with the fontenc.

\usepackage{graphicx}
\newcommand{\Def}[1]{{\textbf{#1}}}
\newcommand{\TODO}[1]{\textbf{#1}}
\newcommand{\InsertPicture}[2]
{
	\begin{figure}	
	\includegraphics[scale=#2]{Images/#1}
	\end{figure}
}
\newcommand{\Code}[1]{\texttt{#1}}
\newcommand{\Type}[1]{\emph{#1}}
\newcommand{\RefTable}[1]{{Table \ref{tbl:#1}}}
\newcommand{\BrackType}[1]{$<$\emph{#1}$>$}
\newcommand{\RefPicture}[1]{{{Figure~\ref{fig:#1}}}}
\newcommand{\CodeCaption}[2]{{Example #1} - #2}
\newcommand{\Metatype}[1]{$\overline{#1}$}

\title[MPFL] % (optional, use only with long paper titles)
{A Language-Based Software Framework \\for Mission Planning\\in Autonomous Mobile Robots}


\author[Mirza A. Shah] % (optional, use only with lots of authors)
{Mirza A. Shah}
% - Give the names in the same order as the appear in the paper.
% - Use the \inst{?} command only if the authors have different
%   affiliation.

\institute[Penn State CSE] % (optional, but mostly needed)
{\textbf{Department of Computer Science \& Engineering}\\\textbf{Penn State University}
\\M.S. Thesis Defense\\Advisor: Dr. John Hannan}

% - Use the \inst command only if there are several affiliations.
% - Keep it simple, no one is interested in your street address.

\date[] % (optional, should be abbreviation of conference name)
{July 1, 2011}
% - Either use conference name or its abbreviation.
% - Not really informative to the audience, more for people (including
%   yourself) who are reading the slides online

\subject{Autonomous Robots}
% This is only inserted into the PDF information catalog. Can be left
% out. 

% If you have a file called "university-logo-filename.xxx", where xxx
% is a graphic format that can be processed by latex or pdflatex,
% resp., then you can add a logo as follows:

%\pgfdeclareimage[height=0.25cm]{university-logo}{arl.png}
%\logo{\pgfuseimage{university-logo}}


% If you wish to uncover everything in a step-wise fashion, uncomment
% the following command: 

%\beamerdefaultoverlayspecification{<+->}

\begin{document}

\begin{frame}
  \titlepage
\end{frame}

%Areas of Interest
%**************************************************************************
\begin{frame}
	\begin{itemize}
	\item Autonomous Robots
	\item Artificial Intelligence
	\item Programming Languages
	\item Operations Research
	\item Software Engineering
	\end{itemize}
\end{frame}

%Pictures of Robots
%**************************************************************************
\begin{frame}
\InsertPicture{robotsgalore1.png}{0.4}
\end{frame}

%More robots
%**************************************************************************
\begin{frame}
\InsertPicture{robotsgalore2.png}{0.4}
\end{frame}

%How Autonomous Robots Work
%**************************************************************************
\begin{frame}
	\InsertPicture{sense-model-plan-act.png}{0.35}
\end{frame}

%iRobot Create Schematic
%**************************************************************************
\begin{frame}
	\InsertPicture{creatediagram.png}{0.15}
\end{frame}

%Components of Intelligence
%**************************************************************************
\begin{frame}
	\InsertPicture{components_of_intelligence.png}{0.32}
\end{frame}

%Unified Picture of Intelligence - Combining Sense - Model - Plan - Act
%**************************************************************************
\begin{frame}
\InsertPicture{howrobotworks.png}{0.28}
\end{frame}

%A Problem of Software Engineering
%**************************************************************************
\begin{frame}
	\begin{itemize}
	\item Autonomy is realized in form of software running on robot
	\item More autonomy means robot must be more intelligent
	\item More intelligent means higher robot software complexity
	\item Many tools created to help manage complexity
		\begin{itemize}
		\item Software libraries
		\item Software frameworks
		\item Programming languages
		\item Robot operating systems
		\end{itemize}
	\item Still too much software redundancy
	\end{itemize}
\end{frame} 

%Focus on Planning
%**************************************************************************
\begin{frame}
	\begin{itemize}
	\item Focus on planning
	\item \Def{Planning} is the act of creating a schedule
	\item A \Def{schedule} is a set of timestamped actions that need to be taken to achieve a set of goals
	\end{itemize}
\end{frame}

%Schedule Example
%**************************************************************************
\begin{frame}
\InsertPicture{schedule.png}{0.4}
\end{frame}

%Reactive vs Deliberative Planning
%**************************************************************************
\begin{frame}
\InsertPicture{reactivevsdeliberative.png}{0.4}
\end{frame}

%Current Approaches
%**************************************************************************
\begin{frame}
	\begin{itemize}
	\item \textbf{General-purpose Planners} 
		\begin{itemize}
		\item Take as input goals, constraints, initial state, action/next state pairs
		\item Searches for action sequence to achieve goals
		\item Problem: Search space explosion
		\end{itemize}
	\item \textbf{Domain-specific Planning/Scheduling Algorithms}
		\begin{itemize}
		\item Disparate algorithms for solving specific problems
		\item Often combined together in form of multi-agent systems, resulting in distributed planning system
		\item Studied greatly in fields of operations research, industrial engineering, and logistics
		\item Problem: No unified concept
		\end{itemize}
	\end{itemize}
\end{frame}

% END BACKGROUND
%**************************************************************************
%**************************************************************************
%**************************************************************************

\begin{frame}
	\begin{itemize}
	\item \textbf{My research} - Creating a unified architecture for disparate, domain-specific planning algorithms
	\item \textbf{Why?} - To advance the state of autonomy software
	\begin{itemize}
	\item Increase reusability, modularity, portability
	\item Verification and guarantees of operational correctness, reducing bugs
	\item Build software quicker, cheaper
	\item Make robotics more accessible to outside fields
	\end{itemize}
	\end{itemize}
\end{frame}

%Languages
%**************************************************************************
\begin{frame}
	\begin{itemize}
	\item \textbf{Language} is the foundation of human thought.
	\item Different concepts are easier to reason about in different languages
	\item This is true of programming languages as well...
	\end{itemize}
\end{frame}

%Solving Problems With Languages
%**************************************************************************
\begin{frame}
	\begin{itemize}
	\item \textbf{Think of planning as a language.}
	\item What kind of language would be ideal for describing to a robot what you want it to do?
	\item How do we implement a compiler/interpreter for such a language?
\end{itemize}
\end{frame}

%My Approach - MPFL Intro
%**************************************************************************
\begin{frame}

	\begin{itemize}
	\item \Def{Modular Planning Language \& Framework (MPFL)}
		\begin{itemize}
		\item A \textbf{software framework} for building planning autonomy on a robot.
		\item Robots using MPFL can be controlled by a domain-specific programming language for robotic planning, the \Def{Mission Specification Language (MSL)}.
		\item Users link a client application against MPFL library. Compiler is invoked via API call from client.
		\end{itemize}
	\end{itemize}
	\InsertPicture{mpflhighlevel.png}{0.45}
\end{frame}

%Demo Here!!!
%**************************************************************************
\begin{frame}
\InsertPicture{MPFL_Demo_Block_Diagram.png}{0.4}
\end{frame}

%How MPFL Client interacts with library
%**************************************************************************
\begin{frame}
\InsertPicture{MPFL_client_layers.png}{0.32}
\end{frame}

%Internals of MPFL Demo Client
%**************************************************************************
\begin{frame}
\InsertPicture{MPFL_Demo_Client_Internals.png}{0.26}
\end{frame}

%How MPFL Works
%**************************************************************************
\begin{frame}
\InsertPicture{mpfl_engine_full.png}{0.30}
\end{frame}

%Example Planner Hierarchy
%**************************************************************************
\begin{frame}
\InsertPicture{exampleplannerhierarchy.png}{0.30}
\end{frame}

%Bonus - Plan Instance Context
%**************************************************************************
\begin{frame}
\InsertPicture{planinstanceiscontext.png}{0.26}
\end{frame}

%Bonus - LST State
%**************************************************************************
\begin{frame}
\InsertPicture{ltstate.png}{0.23}
\end{frame}

%Bonus - PI Tree on Initalization
%**************************************************************************
\begin{frame}
\InsertPicture{ltdemooninit.png}{0.36}
\end{frame}

%Bonus - PI Tree after LST Eval
%**************************************************************************
\begin{frame}
\InsertPicture{ltdemoafterlst.png}{0.36}
\end{frame}




%Conclusion
%**************************************************************************
\begin{frame}
	\begin{itemize}
	\item Verification and Correctness
		\begin{itemize}
		\item Static typing and dimensional analysis
		\item Guarantees of operational correctness
		\item Design-by-contract and inability of user to corrupt runtime
		\end{itemize}
	\item Reusability and Modularity
		\begin{itemize}
		\item Build planning autonomy piecemeal in form of plugins
		\item Plugins are reusable across different robots using MPFL
		\end{itemize}	
	\item Simplicity
		\begin{itemize}
		\item Very simple engine, low computational overhead
		\item Easy to learn API and language
		\end{itemize}
	\item Power
		\begin{itemize}
		\item Highly declarative control language (MSL)
		\item Graceful degradation through exception handling
		\item Maps complex deliberative planning problem into a set of independent scheduling problems
		\end{itemize}	
	\end{itemize}
\end{frame}


\end{document}